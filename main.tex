\documentclass[11pt]{article}
\usepackage{amsmath}

\title{Little Book of Rotations}
\author{Patrick Varin}

\begin{document}
\maketitle
\section{Intro}
This handbook on rotations is inspired partially by the Matrix Cookbook\cite{petersen2008matrix}.
\section{Some Operations}
\begin{align}
	[v\times] &=
	\begin{bmatrix}
          0 & -v_3 &  v_2\\
        v_3 &    0 & -v_1\\
       -v_2 &  v_1 &    0\\
	\end{bmatrix}\\
	v\times w &=
	\begin{bmatrix}
          0 & -v_3 &  v_2\\
        v_3 &    0 & -v_1\\
       -v_2 &  v_1 &    0\\
	\end{bmatrix}
	\begin{bmatrix}
		w_1\\
		w_2\\
		w_3
	\end{bmatrix}\\
	\frac{\partial}{\partial w}(v\times w) &= [v\times]\\
	\frac{\partial}{\partial v}(v\times w) &= [w\times]^T = -[w\times]
\end{align}
\section{Quaternions}
\subsection{Basics}
\begin{align}
	% Notation
	q &=
	\begin{bmatrix}
			q_w\\
			q_x\\
			q_y\\
			q_z
	\end{bmatrix}
	=
	\begin{bmatrix}
	q_0\\
	\vec q
	\end{bmatrix}\\
	% Conjugate
 	q^* &=
	\begin{bmatrix}
			q_w\\
			-q_x\\
			-q_y\\
			-q_z
	\end{bmatrix}
	=
	\begin{bmatrix}
		q_0\\
		-\vec q
	\end{bmatrix}\\
 	% Identity
 	q \circ q^* &=
	\begin{bmatrix}
		1\\
		0\\
		0\\
		0\\
	\end{bmatrix}\\
 	% Multiplication Operator
 	[q \circ] &=
	\begin{bmatrix}
		q_0, & -\vec q^T\\
		\vec q & q_0I + [\vec q\times] 
	\end{bmatrix}\\
	&= q_0I +
	\underbrace{\begin{bmatrix}
	0 & -\vec q^T\\
 	\vec q & [\vec q \times]
	\end{bmatrix}}_{\text{skew symmetric}}\\
 	% Multiplication
	q \circ p &=
	\begin{bmatrix}
		q_wp_w - q_xp_x - q_yp_y - q_zp_z\\
		q_wp_x + q_xp_w + q_yp_z - q_zp_y\\
		q_wp_y - q_xp_z + q_yp_w + q_zp_x\\
		q_wp_z + q_xp_y - q_yp_x + q_zp_w
	\end{bmatrix}\\
	&=
	\begin{bmatrix}
		q_0p_0 - \vec q \cdot \vec p\\
		p_0\vec q + q_0 \vec p + \vec q \times \vec p\\
	\end{bmatrix}
\end{align}
	
\subsection{Derivatives}
\begin{align}
 	% Multiplication Derivative 1
	\frac{\partial}{\partial q}(q \circ p) &=
	\begin{bmatrix}
		p_0,    & \vec p^T\\
		-\vec p, & p_0 I + [\vec p\times]^T
 	\end{bmatrix}\\
 	&= [p\circ]^T\\
 	% Multiplication Derivative 2
	\frac{\partial}{\partial p}(q \circ p) &=
	\begin{bmatrix}
		q_0,    & -\vec q^T\\
		\vec q, & q_0 I + [\vec q\times]
 	\end{bmatrix}\\
 	&= [q\circ]\\
 	% Rotation Derivative
 	\frac{\partial}{\partial p}(q \circ p \circ q^*) &= q \circ \frac{\partial}{\partial p}(p \circ q^*)\\
 	&= q \circ
 	\begin{bmatrix}
		q_0,     & \vec q^T\\
		-\vec q, & q_0 I - [q\times]
 	\end{bmatrix}\\
 	&=
 	\begin{bmatrix}
		q_0, & -\vec q^T\\
		\vec q, & q_0I + [\vec q\times] 
	\end{bmatrix}
	\begin{bmatrix}
		q_0,     & \vec q^T\\
		-\vec q, & q_0 I - [q\times]^T
 	\end{bmatrix}\\
 	&= q_0^2I -
 	\begin{bmatrix}
 		||\vec q||^2,& 0\\
 		0,& \vec q \vec q^T - [\vec q \times]^2
 	\end{bmatrix}
\end{align}
\bibliographystyle{ieeetr}
\bibliography{references}
\end{document}
